\documentclass{article}
\usepackage[utf8]{inputenc}
\usepackage{spanish}
\decimalpoint
\title{Sobre el Reporte IPCC}
\author{César Andrés Pérez Robinson}
\date{Enero 2019}

\begin{document}

\maketitle

En este ensayo se abordan los temas expuestos por el reporte del Panel \\ Intergubernamental del Cambio Climático \cite{IPCC}, los impactos observados y proyectados en relación al medio ambiente y una breve introducción a los retos presentes al comunicar tal reporte.

\section{Reporte IPCC}
El reporte se orienta a comunicar e ilustrar los impactos  que se presentarían a partir de un incremento de $1.5^\circ C$ a la temperatura global, haciendo un llamado al público sobre la manera en que se aborda el tema del cambio climático, desarrollo sustentable y esfuerzos por erradicar la pobreza \cite{Allen}.
\\
Se indica que limitar el calentamiento global a un incremento de $1.5^{\circ}C$, aun que posible dentro de las leyes de la química y la física, requerirá un esfuerzo rápido y de gran alcance en cuanto a las emisiones de dióxido de carbono ($CO_{2}$) emitidas por los humanos. 
Las emisiones necesitarían ser reducidas a un 45 por ciento de los niveles del 2010 para el 2030 y mantener una meta de cero emisiones de dióxido de carbono para el 2050.
El enfoque en $1.5^{\circ}C$ fue establecido debido a que la gran mayoría de países aceptaron firmar los Acuerdos de París en Diciembre del 2015, donde establecieron los propósitos de limitar el incremento de la temperatura global a $1.5^{\circ}C$ \cite{Earth}.
\\
Un incremento de $1.5^{\circ}C$ en la temperatura global conlleva una serie de consecuencias ambientales como; sequías, ciclones, mayor nivel del mar, impacto negativo en los corales del mundo, así como consecuencias a los humanos en cuanto a la producción y distribución de alimentos, seguridad nutricional y salud general, entre muchas otras más \cite{Allen}. 

\subsection{Impactos observados y riesgos ambientales proyectados}
El cambio climático tiene un impacto negativo en la disponibilidad del agua y de la seguridad de esta, y se ha concluido que alrededor del 80\% de la población mundial, ya presentan riesgos significativos a la seguridad del agua, debido a su demanda, disponibilidad y contaminación.
\\
El cambio climático tiene un impacto considerable en la disponibilidad del agua, lo que representa un problema significativo para una gran cantidad de islas pequeñas, como lo son las presentes en la región del Caribe, ya que, incluso logrando que la temperatura no suba de $1.5^{\circ}C$, se espera que se presente una reducción del agua dulce, producto del cambio de aridez proyectado.
\\
También, debido a una combinación de temperaturas altas y una reducción de corrientes de río, la capacidad que se puede utilizar de centrales termoeléctricas que usan agua de río para el enfriamiento se espera que reduzcan en los países europeos.
\\
En cuanto al océano, cambios físicos y químicos debido al incremento de $CO_{2}$ y otros gases invernadero ya están presentando cambios significativos en los ecosistemas de los océanos, y seguirán causando dichos cambios a un incremento de $1.5^{\circ}C$. Estos cambios incluyen la acidificación del océano, un incremento en la intensidad de las tormentas y desoxigenación.
\\
Uno de los problemas más conocidos sobre el cambio climático es la pérdida de las capas polares. Ya hay considerable evidencia de que un incremento de $0.5^{\circ}C$ a la temperatura global actual, conlleva múltiples niveles de impacto en una variedad de organismos, desde fitoplancton hasta mamíferos marinos, con los cambios más dramáticos ocurriendo en el Océano Ártico y la Península Antártica.
\\
Otro problema muy popular es el incremento en el nivel del mar, el cual implica impactos sustanciales que ya son percibidos por los ecosistemas y comunidades costales. Los cambios incluyen tormentas de mayor intensidad, daño de estructuras, erosión y pérdida de hábitat. La literatura actual en su gran mayoría concluye que estos impactos se verán intensificados en un incremento de $1.5^{\circ}C$ a la temperatura global, sin embargo, serían incluso mayores a un incremento de $2^{\circ}C$ \cite{Hoegh}.
\subsection{Impactos observados y riesgos en producción alimenticia}
Distintos estudios ya han demostrado el impacto del cambio climático en la estabilidad de las cosechas, lo que resulta en cambios en los niveles de producción de una gran cantidad del trabajo de agricultura. Estos impactos han sido percibidos en distintas áreas del mundo, como Asia, América, Europa y particularmente en aquellas cosechas específicas al clima donde son plantadas, como lo es en el Mediterráneo para cosechas como el olivo.
\\
El curso actual de la temperatura ha reducido la producción y el rendimiento de los cultivos, con los impactos más negativos siendo presentados en el maíz y el trigo. El incremento a la temperatura ha favorecido la producción de cosechas en áreas de altas latitudes. Se estima que el 60\% de la variación en producción del maíz, arroz y haba de soya es producto de la variación climática que se ha presentado en los últimos años.
\\
En cuanto a la producción ganadera, los estudios climáticos han considerado que los impactos son menores, sin embargo, impactará de manera indirecta el sector ganadero debido a la variación en cuanto cantidad y cualidad del alimento, debido al incremento en pesticidas y enfermedades. También se pueden presentar cambios fisiológicos a partir de distrés térmico, sudoración e incremento en la cadencia respiratoria, que se ha observado reducir la producción de leche y un incremento en la mortalidad de las vacas debido al estrés térmico en varias regiones del Reino Unido.
\\
Se espera que el impacto del cambio climático en el ganado incremente mediante la temperatura global crece, reduciendo la calidad del alimento común, además de variaciones significativas debido al cambio en las temporadas de lluvia. De manera global, se espera una disminución en la producción ganadera de un 7 - 10\% en un incremento de $2^{\circ}C$ a la temperatura global, con una pérdida económica asociada de \$9.7 a \$12.6 mil millones.
\\
Mientras tanto, la acuicultura y la pesca contribuyen cerca de 88.6 y 59.8 millones de toneladas de peces y otros productos anualmente y juegan un rol importante en la seguridad alimenticia de un gran número de países. Un incremento de la temperatura tiene grandes implicaciones en el océano, como mayores temperaturas, acidificación y enfermedades. Además, un nivel del mar más alto presenta riesgos a la acuicultura y a la infraestructura requerida \cite{Hoegh}.
\subsection{Impactos observados y riesgos a la salud humana}
Los patrones del cambio climático son asociados con las temporadas y la intensidad de transmisión de seleccionadas enfermedades específicas a las diferentes temporadas, por lo que se ha asociado un incremento en la morbilidad y mortandad a climas extremos.
\\
Se ha concluido que hay una alta confianza en que el cambio climático conlleva a mayor riesgo de accidentes, enfermedades y muerte, causando intensas olas de calor e incendios, incremento en el riesgo de la desnutrición y consecuencias de una reducción a la productividad en poblaciones vulnerables.
\\
Los riesgos proyectados a un incremento de $1.5^{\circ}C$ incluyen enfermedades diarreicas, problemas mentales y una gran cantidad de enfermedades debido a la poca calidad del aire \cite{Hoegh}.
\section{Retos en la comunicación del cambio climático}
Uno de los principales retos en cuanto a la comunicación del cambio climático, está basado en el balance necesario de comunicar dentro de un rigor científico contra un mensaje claro y accesible. Uno de los criticismos dirigidos a los científicos que trabajan sobre cambio climático, suele ser que tienden a ser alarmistas para llamar la atención, por lo que los líderes del Panel Climático, se han enfocado en evitar cualquier cosa que pudiera ser tomada como "sensacional". Mientras que al mismo tiempo, se presenta la cuestión relacionada a que los mensajes importantes de los reportes sean obscurecidos debido a una estilo conservador de lenguaje \cite{Lynn}.
\\
Un reto más, está relacionado al tratamiento de la incertidumbre en cuanto a la comunicación de cuestiones científicas, sobre la cual el reporte del IPCC no se encuentra absuelto. Para los científicos la incertidumbre es considerada normal, sin embargo para aquellos no expertos puede resultar muy confusa. Para el lector lego, el ver la incertidumbre resulta una muestra de lo que no se conoce, mientras que para el científico, expresa aquello que se conoce de una manera conservativa \cite{Lynn}.











\begin{thebibliography}{9}

\bibitem{IPCC}
V. Masson-Delmotte, P. Zhai, H. O. Pörtner, D. Roberts, J. Skea, P.R. Shukla, A. Pirani, W. Moufouma-Okia, C. Péan, R. Pidcock, S. Connors, J. B. R. Matthews, Y. Chen, X. Zhou, M. I. Gomis, E. Lonnoy, T. Maycock, M. Tignor, T. Waterfield, (2018). \emph{Global warming of $1.5^{\circ}C$. An IPCC Special Report on the impacts of global warming of $1.5^{\circ}C$ above pre-industrial levels and related global greenhouse gas emission pathways, in the context of strengthening the global response to the threat of climate change, sustainable development, and efforts to eradicate poverty}. In Press.
\bibitem{Allen}
M. Allen, O. P. Dube, W. Solecki, F. Aragón–Durand, W. Cramer, S. Humphreys, M. Kainuma, J. Kala, N. Mahowald, Y. Mulugetta, R. Perez, M. Wairiu, K. Zickfeld, (2018). \emph{Framing and Context}. In: Global warming of $1.5^{\circ}C$. An IPCC Special Report on the impacts of global warming of $1.5^{\circ}C$ above pre-industrial levels and related global greenhouse gas emission pathways, in the context of strengthening the global response to the threat of climate change, sustainable development, and efforts to eradicate poverty. In Press.

\bibitem{Earth}
Earth Day Network, (2018). \emph{What you need to know about the new IPCC climate report}. https://www.earthday.org/2018/10/08/what-you-need-to-know-
about-the-new-ipcc-climate-report/

\bibitem{Hoegh}
O. Hoegh-Guldberg, D. Jacob, M. Taylor, M. Bindi, S. Brown, I. Camilloni, A. Diedhiou, R. Djalante, K. Ebi, F. Engelbrecht, J. Guiot, Y. Hijioka, S. Mehrotra, A. Payne, S. I. Seneviratne, A. Thomas, R. Warren, G. Zhou, (2018). \emph{Impacts of 1.5ºC global warming on natural and human systems}. In: Global warming of $1.5^{\circ}C$. An IPCC Special Report on the impacts of global warming of $1.5^{\circ}C$ above pre-industrial levels and related global greenhouse gas emission pathways, in the context of strengthening the global response to the threat of climate change, sustainable development, and efforts to eradicate poverty [V. Masson-Delmotte, P. Zhai, H. O. Pörtner, D. Roberts, J. Skea, P.R. Shukla, A. Pirani, W. Moufouma-Okia, C. Péan, R. Pidcock, S. Connors, J. B. R. Matthews, Y. Chen, X. Zhou, M. I. Gomis, E. Lonnoy, T. Maycock, M. Tignor, T. Waterfield (eds.)]. In Press.

\bibitem{Lynn}
Lynn, J. (2018). \emph{Communicating the IPCC: Challenges and opportunities}. In: Handbook of Climate Change Communication. Springer International Publishing. Switzerland.
\end{thebibliography}

\end{document}
https://www.overleaf.com/project/5c465a932fcb3b1393275c00